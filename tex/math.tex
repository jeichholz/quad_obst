\documentclass{article}
\usepackage{amsmath}

\newcommand{\T}{\mathcal{T}}
\renewcommand{\v}{\vec}
\begin{document}
The obstacle problem introduced is to find $u\in K$ such that
\begin{equation}
  \label{eq:obprob}
a(u,v-u)\ge (f,v-f)_\Omega\quad \forall v\in K, 
\end{equation}
where 
\[K=\{v \in H^1_0(\Omega)| v\ge \phi a.e in \Omega.\]

Here $K$ is a closed and convex admissible set of functions and 
\[a(u,v)=\int_\Omega \nabla u \cdot \nabla v\] 
and
\[(u,v)_\Omega=\int_\Omega uv.\]
 

Introduce the functional 
\[E(u)=\frac{1}{2}a(u,u)-(f,u)_\Omega.\]
We can see that \eqref{eq:obprob} is equivalent to finding 
\begin{equation}
  \label{eq:minprob}
  u=\text{argmin}_{v\in K} E(v).
\end{equation}

To see this, consider the Gateaux derivative of $E$ in the
direction of $v$.  We can see that
\begin{align}
  \label{eq:GateauxDeriv}
  &\lim_{t\rightarrow
    0}\frac{E(u+t(v-u))-E(u)}{t}\\
&\quad=\lim_{t\rightarrow
  0}\frac{1/2 a(u+t(v-u),u+t(v-u))-(f,u+t(v-u))-1/2 a(u,u)+(f,u)}{t}\\
&\quad=\lim_{t\rightarrow0}
 \bigg\{1/2\bigg[a(u,u)+ta(v-u,u)+ta(v-u,u)+t^2a(v-u,v-u)\bigg]\\
&\qquad -(f,u)-t(f,v-u)-1/2a(u,u)+(f,u)\bigg\}/t\\
&\quad=\lim_{t\rightarrow
  0}1/2\bigg(a(v-u,u)+a(v-u,u)+ta(v-u,v-u)\bigg)-(f,v-u)\\
&\quad=a(u,v-u)-(f,v-u).
\end{align}

Thus we see that requiring the directional derivative of $E$ at $u$ to
be positive in every possible direction is equivalent to requiring
\eqref{eq:obprob}.

It is not difficult to check that if $u\in H^2$ then $u$ will satisfy

\begin{equation}
  \label{eq:classical-formulation}
  -\Delta u \geq f,\,u\geq \phi,\, (-\Delta u -
  f)(u-\phi)=0,\text{a.e. in } \Omega.
\end{equation}


The natural thing to do for a finite element method is to choose
$K_h\subset K$, and define the numerical solution $u_h$ to be
\[u_h=\text{argmin}_{v_h\in K_h}E(v_h).\]

For the linear finite element method we choose a mesh $\mathcal{T}$
and define  
\[K_h^1=\{v\in H^1_0(\Omega)| v(x_i)\geq \phi(x_i) \text{ at every
  node } x_i \text{ of } \mathcal{T} \text{ and } v|_T\in P_1(T)
\forall T\in \mathcal{T}\}.\]

Suppose that $u_h,v_h\in K_h \subset K$ and that $K$ has basis
$\Phi=\{\phi_i\}_{i=1}^N$.  We have in mind further that $\Phi$ is a
typical finite element basis.  That is, each $\phi_i$ is supported on
only a few triangles $T$ of the mesh $\mathcal{T}$ and there exists a
set of nodes $X=\{x_i\}_{i=1}^N$ such that $\phi_i(x_j)=\delta_{ij}$. 

Given a triangle $T\in \T$ let $X(T)$ be the nodes that lie in $T$. 

Now we may express 
\begin{align}
  &\int_\Omega \nabla u_h\cdot \nabla v_h=\sum_{T\in\T}\int_T\nabla
  u_h\cdot \nabla v_h=\sum_{T \in \T} \int_T \nabla \left(\sum_{i=1}^N
  u_{h,i}\phi_i\right) \cdot \left(\nabla \sum_{j=1}^N v_{h,j}\phi_j\right)\\
  &\quad=\sum_{T\in \T} \int_T \sum_{i=1}^N u_{h,i}\nabla \phi_i \cdot
  \sum_{j=1}^N v_{h,j}\nabla \phi_j.
\end{align}

Now assuming $\nabla \phi_i=0$ if $i\notin X(T)$ we have
\begin{align}
  &=\sum_{T\in \T}\int_T\sum_{i,j\in X(T)}u_{h,i}v_{h,j}\nabla \phi_i
  \cdot \nabla \phi_j=\sum_{T\in\T}\sum_{i,j\in
    X(T)}u_{h,i}v_{h_j}\int_T \nabla \phi_i\cdot \nabla\phi_j.
\end{align}

Assume all the normal reference element stuff.  Then
\begin{align*}
&\partial_p\left(\phi_i(x)\right)=\partial_p\left(\hat{\phi}_i\circ T^{k^{-1}}(x)\right)\\
&\quad=\partial_1\hat{\phi}_i(T^{k^{-1}}(x))\partial_pT^{k^{-1}}_1(x)+\partial_2\hat{\phi}_2(T^{k^{-1}}(x))\partial_pT^{k^{-1}}_2(x).
\end{align*}

Now, for any function $v$ defined on $T$, define 
\[\v v=[v(x_{l_1}),v(x_{l_2})\dots,v(x_{l_M})]\]
where $M$ is the number of nodes lying in $T$ (corresponding to the
degree of the FEM method).  Then
\[\partial_p v(x)=\sum_{i=1}^Mv_i\partial_p\phi_i(x)=D_p(x) \v v\]
where
\begin{align*}
&D_p(x)=[\partial_p\phi_1(x),\partial_p\phi_2(x),\dots,\partial_p\phi_M(x)]\\
&\quad =\left(\partial_pT^{k^{-1}}\right)^T[\nabla \hat{\phi}_1(x)|\nabla\hat{\phi}_2(x)|\dots|\nabla\hat{\phi}_M(x)]
\end{align*}

\end{document}
